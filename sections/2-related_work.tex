\section{Related Work}

\subsection{具身认知理论}

具身认知(Embodied Cognition)理论为VR在教育中的应用提供关键理论视角。该理论主张,人类的认知过程并非独立于身体的抽象心智活动,而是深刻地植根于身体的感官系统以及与物理世界的动态交互之中[2]。大量心理学与认知科学研究证实,许多关键的直觉知识,尤其是关于物理世界运行规律的概念,正是在与环境的互动中形成的[3]。例如,个体在对一个复杂的齿轮系统进行心智推理时,往往会无意识地调用其过去与真实物体互动的身体经验,以此为基础进行心智模拟[4]。具身认知在教育领域的研究普遍认为,主动的身体动作与丰富的多感官信息输入是促进知识深度内化和概念理解的关键要素。虚拟现实(VR)技术的发展,为将具身认知理论应用于教学实践提供了理想的平台。其核心优势在于,能够创造出传统模式难以比拟的互动式、沉浸式实验学习机会,将学习者从知识的被动观察者转变为主动探究者。

在身体动作层面,用户的物理动作包括行走、转身、手部抓握能够与虚拟环境中的交互直接映射。例如,在学习杠杆原理时,学生能够通过手势在力臂的不同位置施加作用力,并直观感受用力大小与物体被撬动程度的关系。这种身体的直接参与使得学习过程与个体的物理经验紧密相连,从而促进对抽象物理概念的理解。

在多感官体验层面,沉浸式学习环境提供视觉、听觉、触觉等多感官感受。例如,在地球科学教育中,、研究利用VR将抽象的气候变化数据转化为可交互的场景。学生可以看见不同情景下的海平面场景,观察代表二氧化碳浓度的虚拟粒子在全球范围内流动和积聚。这种具身化动态视觉体验,相比于观看传统的图表和地图,更能有效提升学习者对气候系统复杂性的理解[12]。此外,一项针对生物化学的VR研究表明,当学生能通过力反馈设备与蛋白质分子进行交互时,他们对于蛋白质折叠这一过程的理解深度和记忆持久性,显著优于无力反馈的对照组[14]。同样,在牙科医学训练中,集成力反馈的VR模拟器能够让学生感受到钻孔时不同牙体组织的阻力差异,极大地提升操作技能真实感和训练效果[15]。

\subsection{沉浸式物理实验}
作为传统实验的有效补充,虚拟实验室通过提供安全、可控的探究环境,其教育价值已获广泛认可。在此之中,VR技术以其独特的沉浸式特性,极大地拓展虚拟实验的潜力,能够将抽象科学概念转化为具身体验。这一优势在中学物理教育中尤为突出,尤其是在处理抽象不可见、或因安全与成本限制而难以在常规实验室中开展的实验时,VR具有重要应用价值。

一个典型的例子是电磁学教学。电场、磁场等概念因其不可见性而极其抽象,是学生学习的难点。研究表明,基于VR的模拟实验允许学生在虚拟空间中“看见”并亲手操纵电场线和磁感线,从而极大地增强他们对这些复杂概念的理解和学习兴趣[2]。同样,在力学领域,VR实验也展现出独特的优势。一款名为“抛物线篮球VR”
的严肃游戏,让学生通过在沉浸式环境中反复调整参数、投掷篮球,直观地体验抛体运动的规律。研究证实,这类游戏化实验能显著提升学生对相关概念的掌握程度[3]。此外,VR还能让学生安全地探索在现实中无法进行的实验,例如前往月球表面体验不同的重力加速度,或进入原子核内部观察其结构,从而极大地激发了学习动机和参与度[4]。

然而,尽管VR在视觉和听觉上具有较高沉浸感,但当前绝大多数VR教育应用在触觉反馈方面存在显著不足,这限制具身学习的发展。力、质量、摩擦和碰撞是物理学的重要组成部分,这些概念与触觉和动觉体验密不可分。目前主流的VR控制器仅能提供简单的震动,无法模拟推拉重物时的阻力、感受不同表面的摩擦力或体验磁铁同极相斥的力觉感受[5]。这种触觉通道的缺失,影响学生对物理概念的直观理解。并且,有研究指出,在某些情况下,VR教学并未比传统视频或图像教学带来更显著的学习增益。

\subsection{有形交互与力反馈}
实体用户界面(Tangible User Interface, TUI)是具身认知理论的重要实践,它允许用户通过直接操作实体物体与虚拟环境进行交互。研究表明,TUI在教育领域能够有效促进学生的参与感,帮助他们通过物理表征来理解抽象概念。例如,在Chao等人开发的传感器增强型虚拟实验室中,学生通过操作一个连接着压力传感器的真实注射器来向虚拟容器中打气,注射器的推拉动作直接对应虚拟世界中气体分子数量的增减。这种设计将触觉交互引入虚拟物理教学,显著增强了学生的学习动机与沉浸感。

力触觉反馈则在有形交互的基础上更进一步,不仅接收用户的物理动作,还向用户施加精确的力、振动或运动,使其在虚拟环境中获得操纵物体的感觉。例如,在浮力实验中,学生使用一个力反馈摇杆,将一个虚拟物块压入水中。当物块浸入水的体积逐渐增大时,学生能通过摇杆感受到浮力。研究证实,这种能够直接感知抽象作用力的学习方式,显著提升了学生的学习成效和概念理解深度。

有形交互与力反馈的结合,能够创造出更为直观自然的具身体验。这种将有形交互装置与实时力反馈结合的模式,将抽象的物理规律转化为一种可触摸、可感知的双向互动,从而最大限度地模拟真实的触觉反馈。



