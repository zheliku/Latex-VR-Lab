% --------------------------------------------------
% part 1:引用格式模板
\documentclass[runningheads]{llncs} % 选择文档类型,这里是Springer的llncs模板

\usepackage[T1]{fontenc} % 设置字体编码,保证特殊字符正常显示:

\usepackage[UTF8]{ctex} % 中文支持

\usepackage{graphicx} % 插入图片所需的宏包
% --------------------------------------------------

\begin{document}

% --------------------------------------------------
% part 2:标题、作者、联系方式等内容

% 作者信息,包括作者名、机构、ORCID等
\author{First Author\inst{1}\orcidID{0000-1111-2222-3333} \and
Second Author\inst{2,3}\orcidID{1111-2222-3333-4444} \and
Third Author\inst{3}\orcidID{2222--3333-4444-5555}}

\authorrunning{F. Author et al.} % 文章页眉显示的作者名

% 机构信息和联系方式
\institute{Princeton University, Princeton NJ 08544, USA \and
Springer Heidelberg, Tiergartenstr. 17, 69121 Heidelberg, Germany
\email{lncs@springer.com}\\
\url{http://www.springer.com/gp/computer-science/lncs} \and
ABC Institute, Rupert-Karls-University Heidelberg, Heidelberg, Germany\\
\email{\{abc,lncs\}@uni-heidelberg.de}}

\maketitle % 生成标题、作者、机构等信息
% --------------------------------------------------


% --------------------------------------------------% part 3:摘要和关键词
\begin{abstract} % 摘要部分,简要介绍论文内容
The abstract should briefly summarize the contents of the paper in
150--250 words.

\keywords{First keyword  \and Second keyword \and Another keyword.} % 关键词
\end{abstract}
% --------------------------------------------------


% --------------------------------------------------
% part 4:引入章节内容
% 引入各个章节的内容,这些章节内容存放在sections目录下
% 每个章节的内容都在一个单独的.tex文件中
% 每个人编辑自己负责的章节,便于管理和修改
% 编译时,LaTeX 会自动将这些章节合并成一个完整的文档
\section{Introduction}
\subsection{A Subsection Sample}
Please note that the first paragraph of a section or subsection is not indented. 

The first paragraph that follows a table, figure,
equation etc. does not need an indent, either.

Subsequent paragraphs, however, are indented.

可以先写中文,最后再翻译成英文

xx % 引入第一章:引言

\section{Related Work}
\subsection{A Subsection Sample}
Please note that the first paragraph of a section or subsection is not indented. 

The first paragraph that follows a table, figure,
equation etc. does not need an indent, either.

Subsequent paragraphs, however, are indented.\cite{luo2020dream}

6666
哈haha

lll

pp
 % 引入第二章:相关工作

\input{sections/3-method.tex} % 引入第三章:方法

\section{User Study}
\subsection{A Subsection Sample}
Please note that the first paragraph of a section or subsection is not indented. 

The first paragraph that follows a table, figure,
equation etc. does not need an indent, either.

Subsequent paragraphs, however, are indented. % 引入第四章:用户研究

\input{sections/5-result.tex} % 引入第五章:结果

\input{sections/6-conclusion.tex} % 引入第六章:结论
% --------------------------------------------------

% --------------------------------------------------
% part 5:参考文献
% 设置参考文献格式
\bibliographystyle{splncs04}

% 引用参考文献数据库
\bibliography{mybibliography}
% --------------------------------------------------

\end{document} % 文档结束